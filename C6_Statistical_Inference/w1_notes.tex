\documentclass[]{article}
\usepackage{lmodern}
\usepackage{amssymb,amsmath}
\usepackage{ifxetex,ifluatex}
\usepackage{fixltx2e} % provides \textsubscript
\ifnum 0\ifxetex 1\fi\ifluatex 1\fi=0 % if pdftex
  \usepackage[T1]{fontenc}
  \usepackage[utf8]{inputenc}
\else % if luatex or xelatex
  \ifxetex
    \usepackage{mathspec}
  \else
    \usepackage{fontspec}
  \fi
  \defaultfontfeatures{Ligatures=TeX,Scale=MatchLowercase}
\fi
% use upquote if available, for straight quotes in verbatim environments
\IfFileExists{upquote.sty}{\usepackage{upquote}}{}
% use microtype if available
\IfFileExists{microtype.sty}{%
\usepackage{microtype}
\UseMicrotypeSet[protrusion]{basicmath} % disable protrusion for tt fonts
}{}
\usepackage[margin=1in]{geometry}
\usepackage{hyperref}
\hypersetup{unicode=true,
            pdftitle={Course 6 - Statistical Inference - Week 1 - Notes},
            pdfauthor={Greg Foletta},
            pdfborder={0 0 0},
            breaklinks=true}
\urlstyle{same}  % don't use monospace font for urls
\usepackage{color}
\usepackage{fancyvrb}
\newcommand{\VerbBar}{|}
\newcommand{\VERB}{\Verb[commandchars=\\\{\}]}
\DefineVerbatimEnvironment{Highlighting}{Verbatim}{commandchars=\\\{\}}
% Add ',fontsize=\small' for more characters per line
\usepackage{framed}
\definecolor{shadecolor}{RGB}{248,248,248}
\newenvironment{Shaded}{\begin{snugshade}}{\end{snugshade}}
\newcommand{\AlertTok}[1]{\textcolor[rgb]{0.94,0.16,0.16}{#1}}
\newcommand{\AnnotationTok}[1]{\textcolor[rgb]{0.56,0.35,0.01}{\textbf{\textit{#1}}}}
\newcommand{\AttributeTok}[1]{\textcolor[rgb]{0.77,0.63,0.00}{#1}}
\newcommand{\BaseNTok}[1]{\textcolor[rgb]{0.00,0.00,0.81}{#1}}
\newcommand{\BuiltInTok}[1]{#1}
\newcommand{\CharTok}[1]{\textcolor[rgb]{0.31,0.60,0.02}{#1}}
\newcommand{\CommentTok}[1]{\textcolor[rgb]{0.56,0.35,0.01}{\textit{#1}}}
\newcommand{\CommentVarTok}[1]{\textcolor[rgb]{0.56,0.35,0.01}{\textbf{\textit{#1}}}}
\newcommand{\ConstantTok}[1]{\textcolor[rgb]{0.00,0.00,0.00}{#1}}
\newcommand{\ControlFlowTok}[1]{\textcolor[rgb]{0.13,0.29,0.53}{\textbf{#1}}}
\newcommand{\DataTypeTok}[1]{\textcolor[rgb]{0.13,0.29,0.53}{#1}}
\newcommand{\DecValTok}[1]{\textcolor[rgb]{0.00,0.00,0.81}{#1}}
\newcommand{\DocumentationTok}[1]{\textcolor[rgb]{0.56,0.35,0.01}{\textbf{\textit{#1}}}}
\newcommand{\ErrorTok}[1]{\textcolor[rgb]{0.64,0.00,0.00}{\textbf{#1}}}
\newcommand{\ExtensionTok}[1]{#1}
\newcommand{\FloatTok}[1]{\textcolor[rgb]{0.00,0.00,0.81}{#1}}
\newcommand{\FunctionTok}[1]{\textcolor[rgb]{0.00,0.00,0.00}{#1}}
\newcommand{\ImportTok}[1]{#1}
\newcommand{\InformationTok}[1]{\textcolor[rgb]{0.56,0.35,0.01}{\textbf{\textit{#1}}}}
\newcommand{\KeywordTok}[1]{\textcolor[rgb]{0.13,0.29,0.53}{\textbf{#1}}}
\newcommand{\NormalTok}[1]{#1}
\newcommand{\OperatorTok}[1]{\textcolor[rgb]{0.81,0.36,0.00}{\textbf{#1}}}
\newcommand{\OtherTok}[1]{\textcolor[rgb]{0.56,0.35,0.01}{#1}}
\newcommand{\PreprocessorTok}[1]{\textcolor[rgb]{0.56,0.35,0.01}{\textit{#1}}}
\newcommand{\RegionMarkerTok}[1]{#1}
\newcommand{\SpecialCharTok}[1]{\textcolor[rgb]{0.00,0.00,0.00}{#1}}
\newcommand{\SpecialStringTok}[1]{\textcolor[rgb]{0.31,0.60,0.02}{#1}}
\newcommand{\StringTok}[1]{\textcolor[rgb]{0.31,0.60,0.02}{#1}}
\newcommand{\VariableTok}[1]{\textcolor[rgb]{0.00,0.00,0.00}{#1}}
\newcommand{\VerbatimStringTok}[1]{\textcolor[rgb]{0.31,0.60,0.02}{#1}}
\newcommand{\WarningTok}[1]{\textcolor[rgb]{0.56,0.35,0.01}{\textbf{\textit{#1}}}}
\usepackage{graphicx,grffile}
\makeatletter
\def\maxwidth{\ifdim\Gin@nat@width>\linewidth\linewidth\else\Gin@nat@width\fi}
\def\maxheight{\ifdim\Gin@nat@height>\textheight\textheight\else\Gin@nat@height\fi}
\makeatother
% Scale images if necessary, so that they will not overflow the page
% margins by default, and it is still possible to overwrite the defaults
% using explicit options in \includegraphics[width, height, ...]{}
\setkeys{Gin}{width=\maxwidth,height=\maxheight,keepaspectratio}
\IfFileExists{parskip.sty}{%
\usepackage{parskip}
}{% else
\setlength{\parindent}{0pt}
\setlength{\parskip}{6pt plus 2pt minus 1pt}
}
\setlength{\emergencystretch}{3em}  % prevent overfull lines
\providecommand{\tightlist}{%
  \setlength{\itemsep}{0pt}\setlength{\parskip}{0pt}}
\setcounter{secnumdepth}{0}
% Redefines (sub)paragraphs to behave more like sections
\ifx\paragraph\undefined\else
\let\oldparagraph\paragraph
\renewcommand{\paragraph}[1]{\oldparagraph{#1}\mbox{}}
\fi
\ifx\subparagraph\undefined\else
\let\oldsubparagraph\subparagraph
\renewcommand{\subparagraph}[1]{\oldsubparagraph{#1}\mbox{}}
\fi

%%% Use protect on footnotes to avoid problems with footnotes in titles
\let\rmarkdownfootnote\footnote%
\def\footnote{\protect\rmarkdownfootnote}

%%% Change title format to be more compact
\usepackage{titling}

% Create subtitle command for use in maketitle
\providecommand{\subtitle}[1]{
  \posttitle{
    \begin{center}\large#1\end{center}
    }
}

\setlength{\droptitle}{-2em}

  \title{Course 6 - Statistical Inference - Week 1 - Notes}
    \pretitle{\vspace{\droptitle}\centering\huge}
  \posttitle{\par}
    \author{Greg Foletta}
    \preauthor{\centering\large\emph}
  \postauthor{\par}
      \predate{\centering\large\emph}
  \postdate{\par}
    \date{2019-11-22}

\usepackage{booktabs}
\usepackage{longtable}
\usepackage{array}
\usepackage{multirow}
\usepackage{wrapfig}
\usepackage{float}
\usepackage{colortbl}
\usepackage{pdflscape}
\usepackage{tabu}
\usepackage{threeparttable}
\usepackage{threeparttablex}
\usepackage[normalem]{ulem}
\usepackage{makecell}
\usepackage{xcolor}

\begin{document}
\maketitle

\hypertarget{intro}{%
\section{Intro}\label{intro}}

Statistical inference is:

\begin{quote}
The generation of conclusions about a population from a noisy sample
\end{quote}

This class is focussed on frequentist styles of statistics.

Full course information is located at
\href{https://github.com/bcaffo/courses/tree/master/06_StatisticalInference}{on
GitHub}. Videos are on
\href{https://www.youtube.com/playlist?list=PLpl-gQkQivXiBmGyzLrUjzsblmQsLtkzJ}{YouTube}.

\hypertarget{probability}{%
\section{Probability}\label{probability}}

Given a random experiment, a probability measure is a population
quantity that summarises the randomness.

Probability takes a possible outcome from an experiment and:

\begin{itemize}
\item
  Assigns it a number between 0 and 1.
\item
  The probability that \textbf{something} occurs is 1.
\item
  The probability of the union of any two sets of outcomes that have
  nothing in common is the \textbf{sum} of their respective
  probabilities.
\item
  The probability that nothing occurs is 0.
\item
  The probability that something occurs is 1.
\item
  The probability that something is 1 minus the probability that the
  opposite occurs.
\item
  The probability of at least one of two (or more) things that cannnot
  simulaneously occur (mutually exclusive) is the sum of their
  respective probabilities.
\item
  If an event A \textbf{implies} the occurence of event B, then the
  probability of A occurring is less than the probability that B occurs.
\item
  For any two events the probability that \emph{at least} one occurs is
  the sum of their probabilities minus their intersection:

  \begin{itemize}
  \tightlist
  \item
    You can't just add probabilities if they have a non-trivial
    interaction.
  \item
    You're adding the intersection twice.
  \end{itemize}
\end{itemize}

\[ P(A \cup B) = P(A) + P(B) - P(A \cap B) \]

\hypertarget{probability-mass-functions}{%
\subsection{Probability Mass
Functions}\label{probability-mass-functions}}

Densities and mass functions for random variables are the best starting
point.

\textbf{Random variable}: the numerical outcome of an experiment, can be
\textbf{discrete} or \textbf{continuous}. Discrete variables have
values, continuous variables have ranges.

A \emph{probability mass function} (PMF) evaluated at a value
corresponds to the probability that a random variable takes that value.
To be a valid PMF:

\begin{itemize}
\tightlist
\item
  It must always be larger than or equal to 0.
\item
  The sum of the possible values the random variable can take has to add
  up to one.
\end{itemize}

It is used for discrete values.

\hypertarget{bernoulli-distribution}{%
\subsubsection{Bernoulli Distribution}\label{bernoulli-distribution}}

\$ X = 0 \$ represents tails and \$ X = 1 \$ represents heads.

\[ p(x) = (\frac{1}{2})^x(\frac{1}{2})^{1-x} \text{ for } x=0,1 \]

This is for a for a fair coin. Let \(\theta\) be the probability of a
head, and \(1-\theta\) be the probability of a tail. We now have:

\[ p(x) = \theta^x (1-\theta)^{1-x} \text{ for } x=0,1 \]

\hypertarget{probability-density-functions}{%
\subsection{Probability Density
Functions}\label{probability-density-functions}}

The probabilit density function is associated with continuous variables.

TO be valid density function:

\begin{itemize}
\tightlist
\item
  It must be larger than or equal to zero everywhere.
\item
  The total area under it must be one.
\end{itemize}

Areas under a PDF correspond to probabilities for that random variable.
The probability of a \emph{specific} value is zero, because the area of
a single line through the density function is zero. The bell curve is
initially difficult to work with, so we'll try a simpler version:

\[ 
f(x) = 
\Bigg\{
\begin{array}{ll}
    2x \text{ for } 0 < x < 1\\
    0 \text{ otherwise}
\end{array}
\]

Imagine this is the proportion of help calls that get addressed by a
help line on any given day. So the probability that between 20\% and
60\% of calls get addressed that day is given by the area between .2 and
.6 on the x-axis, bounded by the slope of the line on the y-axis.

\begin{Shaded}
\begin{Highlighting}[]
\KeywordTok{tibble}\NormalTok{(}
    \DataTypeTok{x =} \KeywordTok{seq}\NormalTok{(}\DecValTok{0}\NormalTok{, }\DecValTok{1}\NormalTok{, }\FloatTok{0.1}\NormalTok{),}
    \DataTypeTok{y =} \DecValTok{2}\OperatorTok{*}\NormalTok{x}
\NormalTok{) }\OperatorTok
\StringTok{    }\KeywordTok{ggplot}\NormalTok{(}\KeywordTok{aes}\NormalTok{(x,y)) }\OperatorTok{+}
\StringTok{    }\KeywordTok{geom_line}\NormalTok{()}
\end{Highlighting}
\end{Shaded}

\includegraphics{w1_notes_files/figure-latex/pdf-1.pdf}

Is this a mathematically valid density? It's always bigger than 0, and
its area is

\begin{Shaded}
\begin{Highlighting}[]
\FloatTok{.5} \OperatorTok{*}\StringTok{ }\DecValTok{1} \OperatorTok{*}\StringTok{ }\DecValTok{2}
\end{Highlighting}
\end{Shaded}

\begin{verbatim}
## [1] 1
\end{verbatim}

So it is valid.

Back to our example, what's the probability that 75\% or fewer ({[}0\%,
75\%{]}) of calls are answered? That's the area under the density
function between 0 and .75.

This is the area of a smaller right angle triangle, so it's:

\begin{Shaded}
\begin{Highlighting}[]
\FloatTok{.5} \OperatorTok{*}\StringTok{ }\FloatTok{.75} \OperatorTok{*}\StringTok{ }\FloatTok{1.5}
\end{Highlighting}
\end{Shaded}

\begin{verbatim}
## [1] 0.5625
\end{verbatim}

This is actually a special case of a beta distribution:

\begin{Shaded}
\begin{Highlighting}[]
\KeywordTok{pbeta}\NormalTok{(.}\DecValTok{75}\NormalTok{, }\DecValTok{2}\NormalTok{, }\DecValTok{1}\NormalTok{)}
\end{Highlighting}
\end{Shaded}

\begin{verbatim}
## [1] 0.5625
\end{verbatim}

\hypertarget{cdf-and-survival-function}{%
\section{CDF and Survival Function}\label{cdf-and-survival-function}}

The \textbf{cumulative distribution function} of a random variable,
\(X\), returns the probability that the random variable is less that or
equal to the value \(x\):

\[ F(x) = P(X \le x) \]

This applied whether the variable is discrete or continuous.

The \textbf{survival function} of a random variable \(X\) is defined as
the probability that the random variable is greater than the value
\(x\):

\[ S(x) = P(X > x) \]

Notice that:

\[ S(x) = 1 - F(x) \]

\hypertarget{quantiles}{%
\section{Quantiles}\label{quantiles}}

The \(\alpha\)th quantile of a distribution function \(F\) is the point
\(x_\alpha\) so that \(F(x_\alpha) = \alpha\).

A percentile is simply a quantile with \(\alpha\) expressed as a
percent. The median is the 50th percentile.

Let's find the median for our previous example:

\begin{Shaded}
\begin{Highlighting}[]
\KeywordTok{qbeta}\NormalTok{(.}\DecValTok{5}\NormalTok{, }\DecValTok{2}\NormalTok{, }\DecValTok{1}\NormalTok{)}
\end{Highlighting}
\end{Shaded}

\begin{verbatim}
## [1] 0.7071068
\end{verbatim}

This tells us that on - On 50\% of the days, 70\% or less of phone calls
are answered. - On 50\% of the days, 70\% or more of phone calls are
answered.

It's important to note that here we're talking about the
\textbf{population quantiles}, therefore the median being discussed is
the \textbf{population median}.

The median most people think of is the \textbf{sample median}, taking a
sample, ordering them and taking the middle observation.

We need to look at estimators and estimands. Sample median estimating
the population median. There's assumptions that connect the sample to
the population but we're going to formally develop them.

\textbf{This is the formal process of statistical inference}. Linking
the sample to a population.

\hypertarget{conditional-probability}{%
\section{Conditional Probability}\label{conditional-probability}}

The probability of rolling a one on a die is one in six. If you're given
extra information that someone rolled the die and it was a one, three or
five, conditional on this new information you'd now say the probability
is one in three.

The definition:

\begin{itemize}
\tightlist
\item
  Let \(B\) be an event so that \(P(B) > 0\).
\item
  Then the probability of A given the event B has ocurred is the
  probability of the intersection of A and B, divided by the probability
  of B
\end{itemize}

\[ P(A|B) = \frac{ P(A \cap B) }{ P(B) }\]

In the event that A and B are unrelated, then \(P(A|B) = P(A)\).

Taking our previous die roll example, \(A = \{1\}, B = \{1,3,5\}\)

\[
P(\text{one given roll is odd}) = P(A|B) \\
= \frac{P(A \cap B)}{P(B)} \\
= \frac{P(A)}{P(B)} \\
= \frac{1/6}{1/3} \\
= \frac{1}{3}
\] Remembering that A is containted within B, hence the intersection is
simply A.

\hypertarget{bayes-rule}{%
\subsection{Bayes' Rule}\label{bayes-rule}}

Bayes' Rule allows us to reverse the role of the conditioning set and
the set we want the probability of. We want P(B\textbar{}A) when we have
or easily can calculate P(A\textbar{}B). Bayess rule can do it:

\[ P(B|A) = \frac{
    P(A|B)P(B)
}{
    P(A|B)P(B) + P(A|B^C)P(B^C)
}\]

Where \(P(A|B^c)\) is the probability of B \textbf{marginalised} over A.

\hypertarget{diagnostic-tests}{%
\subsubsection{Diagnostic Tests}\label{diagnostic-tests}}

\begin{itemize}
\tightlist
\item
  Let \(+\) and \(-\) be the events that the results of a diagnostic
  test is positive or negative.
\item
  Let \(D\) and \(D^c\) be the event that the subject of the test has or
  does not have the disease respectively.
\end{itemize}

\[ Sensitivity = P(+ | D) \] This is the marker of a good test, you want
the sensitivity to be high.

\[ Specificity = P(-|D^c) \]

The specificity is the probability the test is negative given that the
subject does not have the disease. Again you want the specificity to be
high for the test to be good.

If you have a positive test, the number that is of most concern to you
is

\[ P(D|+) \]

This is the positive predictive value.

If you have a negative test, you are interested in:

\[ P(D^c|-) \] This is the negative predictive value.

In the absenve of a test

\[ P(D) \]

is the prevalence of the disease.

Let's take a made up example:

\begin{itemize}
\tightlist
\item
  Sensitivity of a test: 99.7\%
\item
  Specificity of a test: 98.5\%
\item
  Prevalence of disease: .01\%
\end{itemize}

We can plug directly into Bayes' rule. We note that the probability of a
postive result, given that the person does not have the disease, is one
minus the probability of a negative result given that the person doesn't
have the disease: \(P(+|D^c) = 1 - P(-|D^c)\). Similarly
\(P(D^c) = 1 - P(D)\)

\$\$ P(D\textbar{}+) = \frac{
    P(+|D) P(D)
}{
    P(+|D)P(D) + P(+|D^c)P(D^c)
} \textbackslash{} = \frac{
    .997 * .001
}{
    .997 * .001 + .015 * .999
} \textbackslash{}

\$\$

\begin{Shaded}
\begin{Highlighting}[]
\NormalTok{(.}\DecValTok{997} \OperatorTok{*}\StringTok{ }\FloatTok{.001}\NormalTok{) }\OperatorTok{/}\StringTok{ }\NormalTok{((.}\DecValTok{997} \OperatorTok{*}\StringTok{ }\FloatTok{.001}\NormalTok{) }\OperatorTok{+}\StringTok{ }\NormalTok{(.}\DecValTok{015} \OperatorTok{*}\StringTok{ }\FloatTok{.999}\NormalTok{))}
\end{Highlighting}
\end{Shaded}

\begin{verbatim}
## [1] 0.06238268
\end{verbatim}

So the positive predictive value is 6\% for this test. The low positive
predicive value is largely due to the low prevalence of the disease.

However imagine if there were other factors about the person the test
was conducted on. The relevant prevalence may then be raised or lowered
based on these factors. An example would be in an HIV test that the
person let it be known they were an intravenous drug user. The
prevalance has changed from the entire population to intravenous drug
users.

\hypertarget{likelyhood-ratios}{%
\subsubsection{Likelyhood Ratios}\label{likelyhood-ratios}}

Note that in both \(P(D|+)\) and \(P(D^c|+)\), the denominator is
exactly the same. If we divide the two equations we get:

\[ \frac{ P(D|+) }{ P(D^c|+) } = \frac{ P(+|D) }{ P(+|D^c) } \times \frac{ P(D) }{ P(D^c) }\]

Whenever you take a probability and divide it by one minus that
probability you get the odds (this is the first item above).

On the left most side we have the odds of disease given a positive test
result. On the right we have the odds of disease in the absence of the
test result.

The factor in the middle is the diagnostic likelihood ratio of a
positive test result. It is \emph{the factor by which you multiply your
odds in the presence of a positive test to obtain your post-test odds.}.

\hypertarget{independence}{%
\section{Independence}\label{independence}}


\end{document}
